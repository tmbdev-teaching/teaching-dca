\documentclass[11pt]{article}
\usepackage{times}
\usepackage[latin1]{inputenc}
\usepackage{ifthen}
%This is documentsubstyle DINA4 for DIN A4 pagesize.
  \topmargin 0mm
  \oddsidemargin 5mm
  \evensidemargin 5mm
  \textwidth 150mm
    % mods for 10 pt (\baselineskip=12pt)
    \textheight 634 pt  % = 222.5 mm
%% mods for 11 pt (\baselineskip=13.6pt)
%    \textheight 635.601 pt % = 223 mm  (!Rundungsfehler!)
%% mods for 12 pt  (\baselineskip=14.5pt)
%    \textheight 633.5 pt % = 222.8mm
 
\marginparwidth 0mm
\marginparsep 0mm
\marginparpush 0pt
\columnwidth\textwidth

\newcommand{\Veranstaltung}{Document~and~Content~Analysis~(SS~2009)}
\renewcommand{\thepage}{{\small\Veranstaltung}\hfill(\arabic{Blattnr})~\arabic{page}/\thelastpagenumber}
\makeatletter\def\thelastpagenumber{\textbf{??}}\AtEndDocument{\immediate\write\@auxout{\string\global\string\def\string\thelastpagenumber{\arabic{page}}}}\makeatother
\newcounter{Blattnr}
\newcommand{\AufgName}{Exercise}
\newtheorem{Aufg}{\AufgName}[Blattnr]
\newenvironment{Aufgabe}[2][]{%
 \ifthenelse{\equal{#1}{}}{\begin{Aufg}}{\begin{Aufg}[#1]}\normalfont\mbox{}#2\\
}{%
 \end{Aufg}%
}
\newenvironment{Aufgabenblatt}[3]{%
 \newpage%
 \setcounter{page}{1}%
 \setcounter{Blattnr}{#1}%
 \noindent
 Image~Understanding~and~Pattern~Recognition \hfill #2\\
 Prof.~Dr.~Thomas~M.~Breuel \hfill Dr.~Faisal~Shafait\\[2\baselineskip]
 \centerline{\textbf{\Large \Veranstaltung}}\\[\baselineskip]
 \centerline{\textbf{\Large\Vorsatz Exercise Sheet \theBlattnr}}\\[2\baselineskip]
 \AbgabeText{#3}\\
% \footnotesize
% \Rule
% Bearbeiten Sie zur Vertiefung der Vorlesung nach M�glichkeit stets alle Aufgaben!\\
% Mit Punkten ausgezeichnete Aufgaben k�nnen in den �bungen zur Korrektur abgegeben werden.\\
% Bitte kommentieren Sie ihre L�sungen dann stets ausf�hrlich!
% \\[-1.1ex]
 \Rule
 \par
 \normalsize
}{%
 \par
 \vfill
%%% \vspace{1ex}
 \noindent
 \Rule
 \texttt{\footnotesize http://courses.iupr.org}
 \par
}
\newcommand{\AbgabeText}[1]{\textbf{to be submitted by E-Mail to faisal.shafait@dfki.de by: #1}}
\newcommand{\Vorsatz}{}
\newcommand{\Vorlaeufig}{\fbox{Vorl�ufiges (!)} }
\newcommand{\Rule}[1][0pt]{\rule{\linewidth}{.5pt}\\[#1]}
\newcommand{\Punkte}[1]{\hfill(#1~Point\ifnum#1=1\else s\fi)}
\newcounter{Enumi}
\newcommand{\abccontinue}{\arabic{enumi}}
\newenvironment{abcenumerate}[1][0]{%
 \ifnum1=1#1\setcounter{Enumi}{\value{enumi}}\else\setcounter{Enumi}{#1}\fi%
 \begin{enumerate}\setcounter{enumi}{\value{Enumi}}\renewcommand{\labelenumi}{\alph{enumi})}%
}{%
 \end{enumerate}%
}
\newcommand{\abcenumerateUnskip}[1][-1.2\baselineskip]{\vspace{#1}}
\def\$#1${\mbox{$#1$}}

\usepackage{amsmath}
\begin{document}
\begin{Aufgabenblatt}{1}{29.04.2009}{04.05.2009}


For most of the exercises, we will be using Python. To get started with Python,
please look at the Python tutorial: http://docs.python.org/tutorial/


%\clearpage
\begin{Aufgabe}[Exploiting data redundancy for compression]{}
  Which types of redundancy are present in digital audio signals and how are these
  used for the compression of audio files in MP3 format? Please answer
  with a short text of about 200 words.
\end{Aufgabe}

\begin{Aufgabe}[Run Length Encoding and Decoding]{}
  Write a program in Python that reads binary data from a file (see binary-data.txt as
  an example) and encodes that in a run-length representation. What compression
  ratio do you get? Save this to a file with .rle extension.\\
  Write another program that decodes the run-length encoded file written by the
  first program and restores the original data.
\end{Aufgabe}

\begin{Aufgabe}[Discrete Cosine Transform]{}
  Write a program in Python that reads an 8x8 matrix from a file (see matrix.txt as
  an example) and computes a 2D DCT of that matrix.
\end{Aufgabe}


\end{Aufgabenblatt}
\end{document}

\end{Aufgabe}

